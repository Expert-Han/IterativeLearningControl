
%-------------------------------------------------------------------------------
\chapter{Einleitung}
%-------------------------------------------------------------------------------

Die Einleitung gliedert sich in den 3 Unterpunkten:

%-------------------------------------------------------------------------------
\section{Motivation} 
%-------------------------------------------------------------------------------

Die Einleitung muss den Leser zum weiterlesen bewegen. Hilfreich sind hier manchmal konkrete Beispiele für Anwendungen.
Die Problemstellung motivieren, erklären warum die Problemstellung behandelt wird und wofür eine Lösung des Problems wichtig ist.

%-------------------------------------------------------------------------------
\section{Zielsetzung} 
%-------------------------------------------------------------------------------

Motivieren Sie Ihr Thema und geben Sie deutlich die Problemstellung an!
Formulieren Sie zu Beginn der Arbeit die Forschungsfrage/n, die Sie in der Arbeit beantworten wollen. In der Regel haben Sie eine Hauptfrage. Zur Unterstützung können dann verschiedene Teilfragen formuliert werden.

%-------------------------------------------------------------------------------
\section{Aufbau der Arbeit} 
%-------------------------------------------------------------------------------

Am Ende der Einleitung muss die Struktur der Arbeit kurz erläutert werden.
Die Arbeit ist in $x$ Kapiteln aufgebaut. Im ersten Kapitel wird die Problemstellung ausführlich formuliert. Außerdem werden die in der Literatur vorhandene Verfahren zur Lösung des Problems diskutiert.

Das zweite und dritte Kapitel bilden zusammen der Kern dieser Arbeit. Dort wird das Lösungsansatz vorgestellt und erklärt.....
Im letzten Kapitel werden die vorgestellten Algorithmen und Ansätze durch Simulationsergnisse validiert....



%\begin{figure}[h]
%	\centering
%		\includegraphics[width=0.65\textwidth]{Grafik11}
%	\caption{Beispiel 1}
%	\label{fig:Grafik11}
%\end{figure}

\newpage
%-------------------------------------------------------------------------------
\chapter{Stand der Technik}
%-------------------------------------------------------------------------------

Es ist wichtig, dass die Literatur studiert wird. In der Literatur angebotene Lösungsansätze sollten kurz diskutiert und besprochen werden und die Wahl des Verfahrens für die Bachelorarbeit sollte begründet werden.
%Erklären, wie bisher an die Problemstellung herangegangen wurde und wichtige vorhandene Lösungsansätze kurz erklären und diskutieren (Vor und Nachteile).\\
\newpage

%-------------------------------------------------------------------------------