\chapter{Basics} 
\label{ChGrDef}
{\color{red}$<$Einfuehrung$>$}

We consider a  discrete time iteration
\begin{align}
\label{eq:GP}
&\begin{array}{c c c c c}
x(t+1) &= &A x(t)& + &B u(t)  \\
y ( t) &= &C x(t)&  + &D u(t),\\ 
\end{array}\\
&\; \;\;x(0) =x_0 , \; \;  t \geq 0,
\end{align}
 with error 
\begin{align}
\label{eq:error}
e(t) =r(t) - y(t)
\end{align}
at time $t  = 0, 1, 2, \dots$.

Here $A \in \mathbb{R}^{n\times n}$, $B \in \mathbb{R}^{n \times l}$, $C \in \R^{m \times n}$, and $D \in \R^{m\times l}$ are real matrices, $r(t) \in \R^{m}$ is a reference signal at time $t$. For notation simplification, we write $(A, B,C,D)$ and mean it to be the discrete-time control system \eqref{eq:GP}.



We aim to find a ''good'' input signal $u(t)$, such that the reference $r(t)$ is tracked possibly well for all $t \geq 0$. With other words, it means, that the error norm $||e(\cdot)||$ must stay small in some  $||\cdot||$, e.g. the $L_2$-norm.
We can achieve it by finding a stabilizing controller -- such that our system state goes not to infinity over time $t$. 
To design it, we must first clearly define what stability means. Since let us first formulate the common definitions and entertain ideas of a stabilizing controller design. 

\section{Definitions}

\begin{defi}
	A discrete time dynamical system
	\begin{align}
	\label{eq:x=Ax}
	x(t+1) = A x(t)
	\end{align}
	for $t \geq 0$ 
	is said to be asymptotically stable, if for all initial condition $x(0) \in \R^n$
	\begin{align}
	\lim_{t \to \infty} x(t) = 0
	\end{align}	
    is satisfied. 
	It is said to be stable, if 
	\begin{align}
	\limsup_{t \to \infty} ||x(t)|| < \infty
	\end{align}
	for all	 initial condition $x(0) \in \R^n$.
	We call the matrix $A$ (asymptotically) stable, if the dynamical system \eqref{eq:x=Ax} is (asymptotically) stable.
\end{defi}


%\begin{theo}
%	A discrete time dynamical system \eqref{eq:x=Ax} is
%	asymptotically stable, if and only if each eigenvalue $\lambda$ of the matrix $A$ lies within the unit circle: $\lambda \in \{x \in \R: \; |x|<1\}$.
%	
%%	It is stable if for each eigenvalue $\lambda$ is satisfied: 
%%	\begin{align}
%%	\label{eq:stability}
%%	\begin{split}
%%	&\lambda \in \{z \in \C: \; |z|\leq 1\} \text{ and for all } \lambda \text{ with } |\lambda| = 1 \\
%%	&\text{ the Jordan Block \cite{LAAG} of the eigenvalue has dimension 1}. 
%%	\end{split} 
%%	\end{align}
%\end{theo}
%\begin{proof}
%	The solution of \eqref{eq:x=Ax} is 
%	\begin{align}
%	x(t) = A^t x_0.
%	\end{align}
%	
%	With Jordan Normal Form we can find a matrix $T \in \GL_n(\R)$, and triangular matrix $\Lambda$, such that $\Lambda$ has the eigenvalues of $A$ on its diagonal and the following is fulfilled: 
%\begin{align}
%	A &= T^{-1} \Lambda T ,\\
%	A^t &= T^{-1} \Lambda^t T. 
%\end{align}
%
%Since all the eigenvalues of $A$ are smaller then one in its absolute value, $\Lambda^t \to 0$ for $t \to \infty$, and hence 
%\begin{align}
%\lim_{t \to \infty}x(t) =\lim_{t \to \infty} A^t x_0 =\lim_{t \to \infty} T^{-1} \Lambda^t T x_0 = 0.
%\end{align}
%	
%\end{proof}
%
%We want to have an easy criteria to check, if a system is stable. This is possible, by examine the eigenvalues of the matrix $A$. 
%But we formulate a more restrictive criteria, which is rather usable for us. 
%%The less strong criteria can be defined via spectral norm
%%\begin{align}
%%\label{eq:specNorm}
%%||A||_\lambda = \sigma_{\max}(A) = \sqrt{\lama (A^*A)} \text{ for } A \in \C^{m\times n}.
%%\end{align}
%%Here $A^*$ means complex conjugate transpose of matrix $A$, and $\lama$ is the maximal eigenvalue.
%
%\begin{theo}
%	\label{thm:||A||<1}
%	Let $||\cdot||: \C^{m \times n} \to \R^+$ be a consistent matrix norm . 
%	
%	A discrete dynamical system \eqref{eq:x=Ax} is asymptotically stable if the norm of the matrix $A$ is less unit: 
%	\begin{align}
%		||A|| <1.
%	\end{align}
%	It is stable, if 
%	\begin{align}
%	||A|| \leq 1.
%	\end{align}
%\end{theo}



In the same way we want to define (asymptotic) stabilizability and detectability. 
\begin{defi}
A a discrete time system $(A, B, C, D)$ or the pair $(A, B)$ is (asymptotically) stabilizable, if there exists a matrix $F$, such that $A - BF$ is (asymptotically) stable.
The system or the pair $(A,C)$ is detectable, if $(A^T, C^T)$ is assymptotically stabilizable. 
\end{defi}



\section{Stabilizing Controller based on Separation Principle}
\label{sch:stabilizingController}



We can find a stabilizing controller for \eqref{eq:GP} using feedback control.
First let us consider the case $y(\cdot) = x(\cdot)$.

%It is very important, that we work with the (asymptotically) stable systems: if $||A||\geq 1$, the solution $x(\cdot)$ will diverge  for $t \to \infty$. 
%
%
%It is also important for finite time horizon: as we have seen before, the supervector description contains terms $A^j$ of power $j = 0, 1, 2, \dots , N-1$. For large $N$ the elements of the matrix $G$ become very large (in its absolute value) and are numerically not achievable. 

Then the system \eqref{eq:GP} becomes stable, if we choose $u(t) = -Fx(t)$ with some matrix $F$, such that $(A - BF)$ is stable.

Clearly, if the system is (asymptotically) stabilizable, we can find such matrix $F$. 

But what is a ''good'' choice for $F$? One option is the so-called Linear-Quadratic-Regulator (LQR), which minimizes the cost function 
\begin{align}
\label{eq:costFcn}
V(u(\cdot)) = \sum_{t = 0}^\infty \left( (x(t))^T Q (x(t)) + u(t)^T R u(t)\right),
\end{align}
overall stabilizing piecewise constant input signals $u(\cdot) \in \R^m$, and with $x(\cdot)$ satisfying \eqref{eq:GP} (with $C = I$ and $D = 0$). 
$Q \succ 0$ and $R \succ 0$ are here some weighting matrices, and allow us  to weight elements in input and output signals. For example, one of the inputs can impact the system as another, or one output must be driven faster. 

The solution of this optimal control problem can be found by choosing 
\begin{align}
\label{eq:Basics:F}
F = (R + B^T P B)^{-1}B^TPA.
\end{align}
where $P$ is a unique symmetric positive definite matrix, the stabilizing solution of the discrete time algebraic Riccati equation
\begin{align}
 A^TA - A^T P B (R + B^T P B)^{-1} B^T P A + Q - P = 0.%TODO: QUELLE!!!!!
\end{align}
Stabilizing solution means, that from \eqref{eq:Basics:F} resulting matrix $(A - BF)$ is stable, and hence this solution stabilizes our system. 




 
%We set 
%\begin{align}
%\label{eq:Q=(Q(t)...),R=(R(t)...)}
%Q = \diag\left(Q(0), Q(1), \dots, Q(N)\right) \text{ and }R = \diag\left(R(0), R(1), R(2), \dots, R(N)\right).
%\end{align}



%The solution of this problem is given by choosing
%\begin{align}
%u(t) = -F(t) x(t),
%\end{align}
%where 
%\begin{align}
%F(t) = (R +B^TP(t+1)B)^{-1}(B^TP(t+1)A).
%\end{align}
%$P(t)$ is found iteratively by solving dynamic Riccati equation: 
%\begin{align}
%\begin{split}
%P(t-1) &= A^TP(t) A - (A^T P(t) B)(R + B^TP(t)B)^{-1}(B^T P(t)A) + Q,\\
%P(N) &= Q.
%\end{split}
%\end{align}
%{\color{red} Quelle? Wiki sagt Chow, Gregory C. (1986). Analysis and Control of Dynamic Economic Systems. Krieger Publ. Co. Das Buch fand ich aber nicht.} 

%The other way to find an optimal $u(\cdot)$ goes via supervectors: 
%
%Then the cost function $V$ can be rewritten as 
%\begin{align}
%V = ||x||_Q^2 + ||u||_R^2,
%\end{align}
%with norms $||\cdot||_Q$ and $||\cdot ||_R$ associated with scalar products 
%\begin{align}
%\label{eq:SkPrQR}
%\langle y,z\rangle_Q = \sum_{t = 0}^N y(t)^TQ(t)z(t), \text{ and } \langle u,v\rangle_R = \sum_{t = 0}^N u(t)^T R(t) v(t).
%\end{align}
%
%
%Then the optimal solution is given \cite{ILCBook}, pp. 104-105, via 
%\begin{align}
%u = G^*x, \text{ where } G^* =R^{-1} G^T Q. 
%\end{align}

For the case $y(\cdot) \neq x(\cdot)$, we can additionally construct an observer to approximate the signal $x(\cdot)$. The  controller we get is said to be based on \textit{separation principle}.
It requires the pair $(A,C)$ to be asymptotically detectable 
which allows us to find a matrix $J$, such that $A - JC$ is asymptotically stable.

We set 
\begin{align}
u= -F \hat{x}, \text{ with } \hat{x} \text{ given via dynamical system }
\end{align}
\begin{align}
\hat{x}(t+1) &= A\hat{x}(t) + B u(t) + J (y - \hat{y}(t)), \\
\hat{y}(t) & = C\hat{x}(t) + D u(t), \\
\hat{x}(0)& = x_0, \; t = 0,1,2, \dots , N.
\end{align}
With this observer system the estimation error can be calculated as \begin{align}
\tilde{x}(t+1) = x(t+1) - \hat{x}(t+1) = (A - JC) \tilde{x}(t)
\end{align} for $t \geq 0$ and hence $\hat{x} (\cdot)$ converges to $x(\cdot)$ at least asymptotically. 

We set $M$ as a right inverse of the matrix 
\begin{align}
D + (C - D F) (I - A + B F)^{-1} B
\end{align}
Then the choice
\begin{align}
u = -F\hat{x} + Mr, 
\end{align}
solves tracking issue for constant $r(\cdot)$. It still can be used for piecewise constant $r(\cdot)$, if the signal does not change too fast.















 
 
 









%
%
%The form \eqref{eq:Gu + d} has a lot of useful properties, some of them we are going get to know in the next section. 
%
%\section{Properties of the Supervector Description} 
%{\color{red}Wird kommen, wenn klar ist, welche Eigenschaften ich sicher brauche. Hier sind die ungefähre }
%
%Let $(A, B, C, D)$ be an $l-$input, $m-$output, state dimension $n$, linear, time-invariant, discrete, state space model of a dynamical system given as 
%\begin{align}
%x(t+1) &= A x(t) + B u (t), \: x(0) = x_0, \: t = 0, \, 1, \, 2, \, \dots, \, N-1\\
%y(t)& = Cx(t) + D u(t), \: t = 0, \, 1, \, 2, \, \dots, \, N. 
%\end{align}
%
%The benefit of the form \eqref{eq:Gu + d} is the possibility to work with the matrices $G$ in a similar way as with transfer matrices:
%
%\begin{theo}{}
%	Let the systems  $(A_1, B_1, C_1, D_1)$ and $(A_2, B_2, C_2, D_2)$ be related in parallel interconnection: they both have the input $u(t)$, and the outputs $\hat{y}(t)$ and $\tilde{y}(t)$ are added to common output $y(t) = \hat{y}(t) + \tilde{y}(t)$. Then the supervector representation of the interconnection is given by 
%		\begin{align}
%		G(A_1, B_1, C_1, D_1) + G(A_2, B_2, C_2, D_2). 
%		\end{align}
%		
%	Let the $r$-input $m$-ouput system $(A_1, B_1, C_1, D_1)$ and $l$-input $r$-ouput system $(A_2, B_2, C_2, D_2)$ be connected in the parallel interconnection. Then the supervector representation of the interconnection is given by 
%	\begin{align}
%	G(A_1, B_1, C_1, D_1)G(A_2, B_2, C_2, D_2). 
%	\end{align}
%\end{theo}
%
%The rank of $G(A,B,C,D)$ is closely related to the rank of $D$ matrix as follows: 
%\begin{theo}
%	\label{thm: relGandK}
%	\begin{enumerate}
%		\item $\ker (G) = \{0\}$ if and only if $\ker(D) = \{0\}$.
%		\item $\im (G) = \R^{m(N+1)}$ if and only if $\im (D) = m$. 
%	\end{enumerate}
%\end{theo} 
%
%
%\section{Noch ein schoenes Section}
%{\color{red} Soll wohl noch irgendein Section kommen}

